\documentclass[aspectratio=169]{beamer}
\usepackage{color,amsmath}
\usepackage{subfigure}
\usepackage{booktabs}
\usepackage{framed}
\usepackage{comment}

\def\vf{\vfill}

%%%%%%%%%%%%%%%%%%%%%%%%%%
\title[]{What are experiments?\\(04-02)}
\author[]{Matthew J. Salganik\\Department of Sociology\\Princeton University}
\date[]{Soc 596: Computational Social Science\\Fall 2016
\vfill
\begin{flushright}
\vspace{0.6in}
\includegraphics[width=0.1\textwidth]{figures/cc.png}
\end{flushright}
}
\begin{document}
%%%%%%%%%%%%%%%%%%%%%%%%%%
\frame{\titlepage}
%%%%%%%%%%%%%%%%%%%%%%%%%%
\begin{frame}

Perturb and observe experiments vs randomized controlled experiments

\end{frame}
%%%%%%%%%%%%%%%%%%%%%%%%%%
\begin{frame}

``It's like you don't harass women, you don't steal, and you've got to have a control group. This is one of the things that you can lose your job for at Harrah's not running a control group.''
Gary Loveman, CEO Harrah's

\pause
\vf
\textcolor{green}{Why is a control group so important?}

\end{frame}
%%%%%%%%%%%%%%%%%%%%%%%%%%
\begin{frame}

Sometimes we create a randomized controlled experiment by creating a treatment group and sometimes we creating a control group

\end{frame}
%%%%%%%%%%%%%%%%%%%%%%%%%%

\end{document}
