\documentclass[aspectratio=169]{beamer}
\usepackage{color,amsmath}
\usepackage{subfigure}
\usepackage{booktabs}
\usepackage{framed}
\usepackage{comment}

\def\vf{\vfill}

%%%%%%%%%%%%%%%%%%%%%%%%%%
\title[]{Logistics\\(03-01)}
\author[]{Matthew J. Salganik\\Department of Sociology\\Princeton University}
\date[]{Soc 596: Computational Social Science\\Fall 2016
\vfill
\begin{flushright}
\vspace{0.6in}
\includegraphics[width=0.1\textwidth]{figures/cc.png}
\end{flushright}
}
\begin{document}
%%%%%%%%%%%%%%%%%%%%%%%%%%
\frame{\titlepage}
%%%%%%%%%%%%%%%%%%%%%%%%%%
\begin{frame}

\begin{itemize}
\item Lab on Wednesday, 2pm - 5pm, Sherrerd 306 (CITP conference room), Bring your own laptop, read Goel et al (2016) before arriving
\pause
\item Prepare for new advice about feedback
\pause
\item A good start for interdisciplinary collaboration is assuming the best about people in other fields
\pause
\item The Heilmeier Catechism
\pause
\end{itemize}

\end{frame}
%%%%%%%%%%%%%%%%%%%%%%%%%%
\begin{frame}

The Heilmeier Catechism (DARPA):\\

\begin{itemize}
\item What are you trying to do? Articulate your objectives using absolutely no jargon.
\item How is it done today, and what are the limits of current practice?
\item What is new in your approach and why do you think it will be successful?
\item Who cares? If you succeed, what difference will it make?
\item What are the risks?
\item How much will it cost?
\item How long will it take?
\item What are the mid-term and final ``exams'' to check for success?
\end{itemize}

\vf

\tiny{\url{https://en.wikipedia.org/wiki/George\_H.\_Heilmeier\#Heilmeier.27s\_Catechism}}

\end{frame}
%%%%%%%%%%%%%%%%%%%%%%%%%%
\begin{frame}

Schedule for this week:
\begin{itemize}
\item Discuss readings (2:10)
\item Break (3:25)
\item Discuss proposals (3:35)
\item Preview of next week (4:35)
\item Introduce lab (4:40)
\item Collect feedback (4:55)
\end{itemize}

\vf
Questions?

\end{frame}
%%%%%%%%%%%%%%%%%%%%%%%%%%

\end{document}
