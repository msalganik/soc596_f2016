\documentclass[aspectratio=169]{beamer}
\usepackage{color,amsmath}
\usepackage{subfigure}
\usepackage{booktabs}
\usepackage{framed}
\usepackage{comment}

\def\vf{\vfill}

%%%%%%%%%%%%%%%%%%%%%%%%%%
\title[]{Non-probability sampling\\(03-03)}
\author[]{Matthew J. Salganik\\Department of Sociology\\Princeton University}
\date[]{Soc 596: Computational Social Science\\Fall 2016
\vfill
\begin{flushright}
\vspace{0.6in}
\includegraphics[width=0.1\textwidth]{figures/cc.png}
\end{flushright}
}
\begin{document}
%%%%%%%%%%%%%%%%%%%%%%%%%%
\frame{\titlepage}
%%%%%%%%%%%%%%%%%%%%%%%%%%
\begin{frame}

\begin{center}
\small{
\begin{tabular}{ l c c c}
           & Sampling & Interviews & Data environment\\
\hline
1st era & Area probability & Face-to-face & Stand-alone \\
2nd era & \parbox[t]{3cm}{\centering Random digital dial\\probability} & Telephone & Stand-alone \\
3rd era & \textcolor{blue}{Non-probability} & Computer-administered  & Linked \\
\end{tabular}
}
\end{center}

\end{frame}
%%%%%%%%%%%%%%%%%%%%%%%%%%%
\begin{frame}

\begin{center}
\includegraphics[width=\textwidth]{figures/prob_vs_nonprob_old}
\end{center}

\vf
\TINY{\url{http://www.chicagotribune.com/news/nationworld/politics/chi-chicagodays-deweydefeats-story-story.html}}

\end{frame}
%%%%%%%%%%%%%%%%%%%%%%%%%%%
\begin{frame}

\vspace{-2.5in}
\begin{center}
\includegraphics[width=\textwidth]{figures/prob_vs_nonprob_new}
\end{center}

\end{frame}
%%%%%%%%%%%%%%%%%%%%%%%%%%
\begin{frame}

\begin{itemize}
\item Probability sample (roughly): every unit from a defined population (a sampling frame) has a known and non-zero probability of inclusion
\pause
\item Not all probability samples are directly representative of the population
\pause
\item But, with appropriate weighting, probability samples can yield unbiased estimates of the frame population
\end{itemize}

\end{frame}
%%%%%%%%%%%%%%%%%%%%%%%%%%
\begin{frame}

\begin{itemize}
\item Key to many adjustment methods is to use external information
\pause
\item If external information is incorrect or used improperly then you can make things worse
\end{itemize}

\end{frame}
%%%%%%%%%%%%%%%%%%%%%%%%%%
\begin{frame}

Imagine that you want to estimate the average height of Princeton students.\\
\begin{itemize}
\item Assume 50\% are male and 50\% are female
\item You stand outside Frist and recruit 60 people
\item Males (n= 20): Average height: 180cm
\item Females (n=40): Average heigh: 170cm
\end{itemize}
\textcolor{green}{What is your estimate of the average height? (think-pair-share at board)}

\end{frame}
%%%%%%%%%%%%%%%%%%%%%%%%%%
\begin{frame}

\begin{itemize}
\item sample mean = 173.3cm ($\frac{180 * 20 + 170 * 40}{20 + 40}$)
\pause
\item weighted estimate = 175cm ($180 * 0.5 + 170 * 0.5$)
\end{itemize}
\pause
\textcolor{green}{How could this go wrong?}

\end{frame}
%%%%%%%%%%%%%%%%%%%%%%%%%%
\begin{frame}

Imagine that you want to estimate the average height of Princeton students.\\
\begin{itemize}
\item Assume 50\% male and 50\% female; assume 25\% first-year; 25\% sophomore; 25\% junior; 25\% senior; assume gender and class year are independent
\item Your (relatively) sample does not include any female seniors.  How could you use the same trick?
\end{itemize}

\end{frame}
%%%%%%%%%%%%%%%%%%%%%%%%%%
\begin{frame}

\begin{center}
\includegraphics[width=\textwidth]{figures/wang_forecasting_2015_title}
\end{center}

\begin{center}
\includegraphics[width=0.5\textwidth]{figures/xboxlogo}
\end{center}

\end{frame}
%%%%%%%%%%%%%%%%%%%%%%%%%%%
\begin{frame}

\begin{center}
\includegraphics[width=0.6\textwidth]{figures/wang_forecasting_2015_fig1}
\end{center}

\begin{itemize}
\item about 750,000 interviews
\item about 350,000 unique respondents
\end{itemize}

\end{frame}
%%%%%%%%%%%%%%%%%%%%%%%%%%%
\begin{frame}

\begin{center}
\includegraphics[width=0.8\textwidth]{figures/wang_forecasting_2015_fig2_and_3}
\end{center}

\end{frame}
%%%%%%%%%%%%%%%%%%%%%%%%%%%
\begin{frame}

\begin{center}
\includegraphics[width=0.9\textwidth]{figures/gelman_buggywhip_blogpost}
\end{center}

\vf
\TINY{\url{http://andrewgelman.com/2014/08/06/president-american-association-buggy-whip-manufacturers-takes-strong-stand-internal-combustion-engine-argues-called-automobile-little-grounding-theory/}}

\end{frame}
%%%%%%%%%%%%%%%%%%%%%%%%%%%
\begin{frame}

\begin{itemize}
\item Mr. P is just one of the many ways to post-stratifiy non-probability samples
\pause
\item the performance of Mr. P (and related methods) is an empirical question
\pause
\item these methods can be applied to big data and experiments
\pause
\item there are also methods that focus on sampling rather than weighting (e.g., sample matching)
\pause
\item we should not let what happened in 1948 prevent us from trying new things today
\end{itemize}

\end{frame}
%%%%%%%%%%%%%%%%%%%%%%%%%%%
\begin{frame}

\begin{center}
\includegraphics[width=0.8\textwidth]{figures/future_sampling}
\end{center}

\end{frame}
%%%%%%%%%%%%%%%%%%%%%%%%%%%


\end{document}
